%%%%%%%%%%%%%%%%%%%%%%%%%%%%%%%%%%%%%%%%%%%%%%%%%%%%
% Document type, global settings, and packages
%%%%%%%%%%%%%%%%%%%%%%%%%%%%%%%%%%%%%%%%%%%%%%%%%%%%

\documentclass[12pt]{report}   %12 point font for Times New Roman
\usepackage{graphicx}  %for images and plots
\usepackage[letterpaper, left=1.5in, right=1in, top=1in, bottom=1in]{geometry}
\usepackage{setspace}  %use this package to set linespacing as desired
\usepackage{times}  %set Times New Roman as the font
\usepackage[explicit]{titlesec}  %title control and formatting
% \usepackage[titles]{tocloft}  %table of contents control and formatting
% \usepackage[backend=bibtex, sorting=none, bibstyle=ieee]{biblatex}  %reference manager
% \usepackage[bookmarks=true, hidelinks]{hyperref}
\usepackage[page]{appendix}  %for appendices
% \usepackage{rotating}  %for rotated, landscape images
% \usepackage[normalem]{ulem}  %for italicized text

\begin{document}

%% Define your thesis title, your name, your school, and your month and year of graduation here

\newcommand{\thesisTitle}{Combinatorial Models for Surface and Free Group Symmetries}
\newcommand{\yourName}{Shane Scott}
\newcommand{\yourSchool}{School of Mathematics}
\newcommand{\yourMonth}{December}
\newcommand{\yourYear}{2018}

\begin{center}

\begin{singlespacing}

\textbf{\MakeUppercase{\thesisTitle}}\\
\vspace{1\baselineskip}
\yourName\\
\vspace{1\baselineskip}
111 Pages\\
\vspace{1\baselineskip}
Directed by Dr. Dan Margalit\\
\vspace{3\baselineskip}
\end{singlespacing}
\end{center}

\doublespacing
\noindent \textbf{Abstract.} The curve complex of Harvey allows combinatorial representation of a surface
mapping class group by describing its action on simple closed curves.
Similar complexes of spheres, free factors, and free splittings allow
combinatorial representation of the automorphisms of a free group.
We consider a Birman exact sequence for combinatorial models of
mapping class groups and free group automorphisms.
We apply this and other extension techniques to compute the
automorphism groups of several simplicial complexes associated
with mapping class groups and automorphisms of free groups.





\end{document}
