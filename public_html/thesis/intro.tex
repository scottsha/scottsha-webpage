% !TEX root = thesis.tex
\chapter{Introduction}




\section{Ivanov's Metaconjecture and Free Group Automorphisms}

This thesis considers two parallel simplicial complexes.
The first is the complex of curves in a surface.
The second is the complex of spheres in 3-space with wormholes.
Both are \emph{combinatorial models} for their respective spaces:
any symmetry of the graph comes from a symmetry of the space itself.


Studies of the mapping class group of a surface make critical use of the curve complex.
The curve complex $\mathcal C S$ has homotopy classes of simple closed curves as vertices and with simplices for disjoint collections of curves.
Harvey first defined the  curve complex $\mathcal C S$ in \cite{MR624817}
 to describe a compactification of Teichm\"uller space and study the mapping class group action.
In the follwing years, the curve complex itself has become \emph{the} space for mapping class group actions.
Harer demonstrated the curve complex is simply connected \cite{MR786348}, and Masur and Minsky showed that it is $\delta$-hyperbolic \cite{MR1714338}, to the delight of Gromov-enthusiasts.
Ivanov showed that the curve $\mathcal C S$ is an exact combinatorial model for
the mapping class group;  the automorphism group of $\mathcal C S$ is the mapping class group \cite{MR1460387}.
The resulting literary explosion of curve-complex rigidity results led Ivanov
to propose his now infamous metaconjecture:

\begin{metaconjecture}
Every sufficiently rich object associated to a surface $S$ has as its group of automorphisms the mapping class group $\mcg^\pm S$.
Moreover, this can
be proved by a reduction to the theorem about the automorphisms of the curve complex $\mathcal C S$.
\end{metaconjecture}



The first statement of the metaconjecture is reasonable, natural, and just vague enough to leave no possibility of gainsay.
The second part proved prophetic, %\footnote{self-fulfillingly so?}
 as the literature has
flourished with rigidity results that rely on the underlying automorphism group of the curve complex.
We refer to Brendle and Margalit for a survey of such results \cite{meta}.
Brendle and Margalit have made a heroic attempt to unify these results in \cite{meta}.
There they show that the class of ``sufficiently rich objects'' includes any normal subgroup of a surface mapping class group containing an element with small support, and any connected simplicial complex of regions in $S$ that does not have pairs of ``exchangable'' vertices.


Less is known about automorphisms of free groups than is known about mapping class groups,
and some promising analogies between the two has translated geometric strategies of Thurston
and others into strategies tackling the algebraic monster $\oout F$ of outer automorphisms of a free group $F$.
Since finite graphs have a free group $F$ as their fundamental group,
one analog suggests considering $\oout F$ as the mapping class of a graph, though homotopy equivalences must substitute for diffeomorphisms.
Just as the mapping class group acts on the Teichm\"uller space of hyperbolic metrics,
Culler and Vogtmann defined an \emph{outer space}  of metrics of graphs \cite{MR830040}.
There are several contenders for a curve complex analog.
Culler-Vogtmann outer space itself retracts to a spine whose simplicial automorphisms are given by  $\oout F$
\cite{vogt}.
Hatcher and Vogtmann suggested the poset of free factors \cite{MR1660045}, and Hatcher also considered the complex of free splittings \cite{MR1314940}.

A more direct geometric analog of the curve complex is given by the complex of spheres in 3-space with wormholes.
By removing open balls of $S^3$ and identifying boundary spheres to form wormholes, we can obtain a manifold $M$ with free fundamental group $F$.
According to
Laudenbach the diffeomorphism of $M$ (up to homotopy) contains $\oout F$ as a finite index subgroup \cite{MR0314054}.
In fact the complex of free splittings of $F$ is naturally isomorphic to the sphere complex $\mathcal S$ of $M$,
with the spheres of $M$ specifying conjugacy classes of splittings of $\pi_1(M)=F$
via
Van Kampen's Theorem.
Aramayana and Souto proved that the automorphism group of the sphere complex $\mathcal S$ is in fact $\oout F$ \cite{souto},
by showing that automorphisms of $\mathcal S$ biject equivariantly to automorphisms of Culler-Vogtmann outer space.
The analog with curves of a surface can be seen more concretely by considering a subsurface of $M$ that tunnels
through the wormholes. In minimal position, spheres are specified by curves of the surface.
This suggests a route to proof of $\oout F$ theorems and an $\oout F$ analog to Ivanov's metaconjecture:
when a proof calls for curves of a surface $S$, consider instead corresponding spheres of $M$.

This will be our major strategy.
Here we advance the goal of an $\oout F$ analog to the Brendle-Margalit theorem, considering the following question:

\begin{question}
  What combinatorial objects associated to a free group $F$ have as their group of automorphisms
  the outer automorphism group $\oout F_n$, and when can this
  be proved by a reduction to the theorem about the automorphisms of Culler-Vogtmann outer space?
\end{question}
The results herein all adhere to this reduction by passing through the complex $\mathcal S$ of spheres in $M_n$.
We consider this a particular incarnation of Margalit and Brendle's generalized metaconjecutre.
\begin{metaconjecture}
  Suppose that $X$ is a nice space. Every object naturally associated to $X$ and having sufficiently
  rich structure has $\oout \pi_1(X)$ as its group of automorphisms.
\end{metaconjecture}

\section{Outline of Results}

The novel technical results presented here are largely $\oout F$ analogs to theorems regarding the curve complex.
We divide these results into three chapters.
Chapter \ref{chap:birman} considers the role of point pushing and the Birman exact sequence in the complex of curves of a surface and in the complex of spheres in $M$.
Chapter \ref{chap:furout} considers some subcomplexes of the sphere complex whose automorphism group is $\oout F$.
Chapter \ref{chap:strongsep} considers low complexity cases for the complex of strongly separating spheres.

\subsection{Birman Point Pushing}

In Chapter \ref{chap:birman}
we consider how adding or removing punctures affects the curve complex or the complex of spheres.
In Section \ref{sect:curvepunc} we reprove the known result
\begin{restatable}{theorem}{thmaddpunc}
  \label{thm:addpunc}
  Let $S_{g,p}$ be the orientable genus $g$ surface with $p$ punctures.
  If the natural map
  $$
  \mcg^\pm S_{g,p} \to  \aaut \mathcal C S_{g,p}
  $$
  is an isomorphism, then so is
  $$
  \mcg^\pm S_{g,p+1} \to  \aaut \mathcal C S_{g,p+1}.
  $$
\end{restatable}
We do so by a new method considering the role of the Birman exact sequence for point pushing in the complex of curves.
The proof follows the following outline.
\begin{enumerate}
  \item For each puncture $q$ there is a puncture-forgetting projection map $\rho_q: \mathcal C S_{g,p+1} \to \mathcal C S_{g,p}$
  that parallels the Birman exact sequence for the mapping class group $\mcg S_{g,p}$,
  so that automorphisms which preserve the fibration of $\rho_q$ must arise from mapping classes.
  \item The fibers of the projection $\rho_q$ are subtrees of $\mathcal C S_{g,p}$,
  with the projection $\rho_q$ collapsing edges between curves that cobound punctured annuli.
  \item The punctured annuli biject to an arc complex of $S_{g,p}$, which we show to be uniquely colored (in the graph-theoretic sense)
  by the punctures of the surface $S_{g,p}$ so that the fibers of the projection $\rho_q$ for various punctures $q$ biject to the
  coloring partition of the arc complex
  \item Automorphisms of $\mathcal C S_{g,p}$ act by automorphism on this arc complex,
  so that the arc complex coloring, and thus the fibers of $\rho_q$, are maintained.
  % \item We conclude that the automorphism group $\aaut \mathcal C S_{g,p}$ satisfies a Birman exact sequence.
\end{enumerate}

The main result of Section \ref{sect:spherepunc} uses an analogous proof.
We show that
\begin{restatable}{theorem}{outpunc}
  The natural map
  $\oout_{n,p} \to \aaut \mathcal S_{n,p}$ is an isomorphism for $n\geq 3$ and $p \geq 0$.
\end{restatable}
where $\oout_{n,p}$ is a relative outer automorphism group and $\mathcal S_{n,p}$ is the complex of spheres in
the punctured manifold $M_{n,p}$ with $n$ ``wormholes'' and $p$ punctures.
The proof is fully analogous to the surface case.
Automorphisms of $\mathcal S_{n,p}$ are shown to respect the fibration of a point-forgetting projection,
so that the proof reduces to considering automorphisms of the sphere complex $\mathcal S$ of $M$.

\subsection{Further Out}

In Chapter \ref{chap:furout}
we consider subcomplexes and associated complexes of the sphere complex.
The main technique of these proofs is to
extend automorphisms from a subcomplex $\mathcal S'$ of $\mathcal S$ to the full sphere complex
by finding a combinatorial characterization of spheres absent from $\mathcal S'$.
Typically this is a \emph{sharing pair} of spheres in $\mathcal S'$
 that intersect in minimal position to
 bound spheres of $\mathcal S$.
 In Section
 \ref{sect:sepspheres} we prove
 \begin{restatable}{theorem}{thmsepspheres}
   The natural map $\oout (F_n) \to \aaut \mathcal S^{sep}_n $
   is an isomorphism for $n \geq 3$.
   \label{thm:sep}
 \end{restatable}
 The proof works by extending automorphism of the separating spheres complex to the nonseparating spheres by observing that
 small separating spheres contain a unique nonseparating sphere.
In Section
 \ref{section:highgenussep}
 we prove
 \begin{restatable}{theorem}{thmhighsep}
   \label{thm:highsep}
   For  $n\geq 3k$,
   the natural map
    $\oout F_n \to \aaut  \mathcal S^{sep,k}_n$
    is an isomorphism.
 \end{restatable}
 where $\mathcal S^{sep,k}_n$ is the complex of spheres freely splitting
 the rank $n$ free group $\pi(M_n)=F_n$ into factors of rank at least $k$.
 The proof is by a sharing pair extension proceeding inductively on the rank $k$.

 In Section \ref{section:ffc}
 we prove the free factor complex $\ffn$ is also a combinatorial model.
 \begin{restatable}{theorem}{bridson}
   For  $n\geq 3$,
   the natural map
    $\oout F_n \to \aaut  \ffn$
    is an isomorphism.
    \label{thm:bridson}
 \end{restatable}


\subsection{Strongly Separating Curves}

In Chapter \ref{chap:strongsep} we consider the
complex of strongly separating curves.
A curve is strongly separating if it is separating but does not bound a twice punctured disk of $S_{g,p}$.
In \cite{bowditch} Bowditch shows that the automorphisms of
$\css S_{g,p}$ are induced by the mapping class group $\mcg^\pm S_{g,p}$ in all but finitely many cases, and asks whether this is true of the remaining few.
In Section \ref{sect:strongpoint} we use the point-forgetting
projection techniques of Chapter \ref{chap:birman} to obtain a few remaining cases.
In Section \ref{sect:leftovers} we give computational evidence for the undecided cases.
